\documentclass[11pt,a4paper]{article}
\usepackage{epsf}
\usepackage{bm}
\usepackage{latexsym}
\RequirePackage{amsopn}
\RequirePackage{amsfonts}
\RequirePackage{amsthm}


\title{p53 in human neural development}
\author{Astrid van der Geest\thanks{a.t.vandergeest@students.uu.nl} and Ana Marin Navarro\thanks{Ana.Marin.Navarro@ki.se}\\ Department of Microbiology, Tumor and Cell Biology\\ Karolinska Institutet, Stockholm, Sweden}

% Here I define a new command "\real" which can now be used in the text
\newcommand{\real}{\mathbb{R}}
% Several other commands that are useful for my purposes
\newtheorem{theorem}{Theorem}
\newtheorem{lemma}{Lemma}
\newcommand{\roo}{\real^{\scriptstyle \Omega\times\Omega}}  %fn space R^Om*Om
\newcommand{\sym}{\mbox{\em sym}}                           %symmetric reps
\newcommand{\alt}{\mbox{\em alt}}                           %alternating reps
\newcommand{\dirsum}{\oplus}                                %direct sum
\newcommand{\comp}{\circ}                                   %composition sign
\newcommand{\sOmega}{{\scriptscriptstyle \Omega}}
\newcommand{\complex}{\mathbb{C}}                           %complex numbers
\newcommand{\CX}{\mathcal{X}}                               %a vector space or category
\newcommand{\ro}{\real^{\scriptstyle \Omega}}               %function space R^\Omega
\newcommand{\sOmegap}{{\scriptscriptstyle \Omega'}}
\newcommand{\CM}{\mathcal{M}}                               %measure category


\begin{document}

\maketitle


\begin{abstract}
Here my abstract will come!
\end{abstract}

\section{Laymen summary \label{intro}}

My laymen summary will put here. 

\section{Introduction \label{intro}}

TP53 is a tumor suppressor gene that is mutated or inactivated in around 50\% of all human cancers \cite{Bouaoun2016}. It is called the guardian of the genome \cite{Lane1992a}. This is because p53 is rapidly upregulated by stress responses like DNA damage \cite{Zilfou2009}. Other stressors are hypoxia, oncogene activation and differentiation cues. Upon cellular stress, p53 elicits responses such as DNA repair, apoptosis or cell cycle arrest. P53 is a part of the p53 family that also contains p63 and p73. They work together to evoke the cellular responses mentioned. 

The tumor suppressor functions of TP53 have been studied extensively to find therapeutic targets for cancer. However, its role during human brain development is less known. P53 is expressed throughout the whole mouse brain during early embryogenesis (Gottlieb, 1997). Additionally, p53 is highly expressed in human embryonic stem cells and gradually decreases expression during cortical differentiation (van de Leemput et al., 2014). The same process holds for murine cortical development with sporadic p53 activity in the adult mouse brain (Schmid, Lorenz, Hameister, \& Montenarh, 1991). 

Studies with the p53 knock out (KO) mice surprisingly reported the mice to develop normally, despite being susceptible to tumors (Donehower et al., 1992). However, further analyses found that between 8 and 23\% of the p53 KO females, developed exencephaly which is an overgrowth of the neural tissue hindering the closure of the neural tube (Armstrong, Kaufman, Harrison, \& Clarke, 1995; Sah et al., 1995). Additionally, one case of a KO male with exencephaly has been reported. When neural stem cells (NSC) are isolated at E13.5 from these mice it has been found that p53 influences NSC proliferation and differentiation (Liu et al., 2013). Loss of p53 stimulates NSC proliferation and directs differentiation of NSC towards the neuronal fate and away from becoming astrocytes. 

/textit{In addition, several studies have shown the importance of maintaining correct levels of p53 protein in the central nervous system (CNS) to avoid inducing aberrant apoptosis and cell cycle arrest [9, 10]. }

Some proteins in the p53 pathway also influence neural development. P53 response gene MDM2 acts as negative feedback by binding p53 for degradation (Haupt, Maya, Kazaz, \& Oren, 1997). Mdm2 KO mice display extensive p53-dependent apoptosis in the ventricular zone of the cerebral cortex, which induces neuroepithelium degeneration (Xiong, Van Pelt, Elizondo-Fraire, Liu, \& Lozano, 2006). This indicates that regulation of p53 during neural development is crucial. 

So far, insight in the role of p53 during human neural development is lacking. Studying early human brain development provides a challenge. Human fetal material is rare and often of low quality. Researchers therefore have relied on mouse models. Indeed, the mouse and human brain have apparent differences: mice have a decreased cortical size as well as a general absence of outer radial glia cells, which are important during human cortical development (Hansen, Lui, Parker, \& Kriegstein, 2010). 

Recent developments in the stem cell biology have opened new windows of studying human brain development. The discovery of cellular reprogramming gives the possibility to derive induced pluripotent stem (iPS) cells derived from human various somatic cells (Takahashi et al., 2007; Takahashi \& Yamanaka, 2006). Neural stem cells and neurons can subsequently be derived from these stem cells. Importantly, a protocol has been developed to human iPS cells for the generation of 3D brain organoids (Lancaster et al., 2013; Lancaster \& Knoblich, 2014). Brain organoids have been shown to almost perfectly match early human brain development and to express key gene pathways operating during human neurodevelopment (Camp et al., 2015; Kelava \& Lancaster, 2016a; Luo et al., 2016). 

In this study, the human iPS cell-derived brain organoid system will be used to study the role of p53 during human brain development. Additionally, we will use a 2D system of neural development by using human neuroepithelial-like stem (NES) cells derived from iPS cells. Since there is a trade-off between complexity of a system and homogeneity of results (Kelava \& Lancaster, 2016b), a combination of a complex 3D and a more controlled 2D in vitro system will allow for more comprehensive analysis of the p53 knock-down (KD) phenotype. With this system, we believe we can aid the understanding of p53's role during early human brain development.  

We found that loss of p53 results in disorganization of the stem cell layer and a delay in generation of TBR1 positive neurons. To further understand p53's role in neural stem cell self-renewal and differentiation we used NES cells. Upon knocking down p53 in NES cells we observed centrosome amplification leading to G2/M-phase arrest, reduced proliferation rate, and chromosomal rearrangements. Gene set enrichment analysis (GSEA) of p53 KD NES cells showed downregulation of DNA damage repair pathways as a possible mechanism for the increase in genomic instability. Interestingly, p53 loss did not impede differentiation of TUJ1 positive neurons. Metabolism result. 


\section{Materials and Methods \label{intro}}

\subsection{Samples}

cells etc. 

\subsection{sections}

Articles are usually structured into sections, and books into chapters and chapter sections. Sections can be cross referenced, if you \verb+\label+ them. For example the Introduction is section \ref{intro} in this document. Sub-sections are possible

\subsubsection{sub-sub-sections are also possible}



\subsection*{subsection without number}




\section{Documentation}




\section{Tables}

Here are some examples of tables.



\begin{center}
\begin{tabular}{rclc}
\multicolumn{3}{c}{subrepresentation} & dim \\
\hline
\multicolumn{3}{c}{\em Diagonal}&\\
$1_D$ &=& ${\scriptstyle \{f~|~f_{ij}=0,~ f_{ii}=c\}}$& ${\scriptstyle 1}$\\
$1^\perp_D$ &=& ${\scriptstyle \{f~|~f_{ij}=0,~ \sum_{i=1}^n f_{ii}=0\}}$& ${\scriptstyle n-1}$\\
\hline
\multicolumn{3}{c}{\em Off-diagonal}&\\
$1_O$ &=& ${\scriptstyle \{f~|~f_{ii}=0,~ f_{ij}=c\}}$ & ${\scriptstyle 1}$\\
$\sym^+$ &=& ${\scriptstyle \{f~|~f_{ij}= \alpha_i+\alpha_j,~ \sum \alpha_i=0\}}$&
${\scriptstyle n-1}$\\
$\sym$ &=& ${\scriptstyle \{f~|~f_{ij}=f_{ji},~ \sum_i f_{ij}=\sum_j f_{ij}=0\}}$&
${\scriptstyle \frac{n(n-3)}{2}}$\\
$alt^+$  &=& ${\scriptstyle \{f~|~f_{ij}= \alpha_i-\alpha_j,~ \sum \alpha_i=0\}}$& ${\scriptstyle n-1
}$\\
$\alt$ &=& ${\scriptstyle \{f~|~f_{ij}=- f_{ji},~ \sum_i f_{ij}=\sum_j f_{ij}=0\}}$&
${\scriptstyle \frac{(n-1)(n-2)}{2}}$
\end{tabular}
\end{center}



\begin{center}
{\small
\begin{tabular}{lll}
Name & Model formula & Conditions \\
\hline
\multicolumn{2}{l}{{\bf diagonal}}\\
$1$ & ${\textstyle \mu_{ii} = \mu} $\\
$\ro$ &${\textstyle \mu_{ii} = \mu + \alpha_i}$ \\
\hline
\multicolumn{2}{l}{{\bf off-diagonal}}\\
{Symmetric}\\
$1$ &${\textstyle \mu_{ij} =  \mu}$\\
$\sym^+$ &${\textstyle \mu_{ij} =  \alpha_i+\alpha_j}$\\
$\sym$& ${\textstyle \mu_{ij} =  \gamma_{ij}}$ &
${\textstyle \gamma_{ij} = \gamma_{ji}}$\\
{Alternating}\\
$\alt^+$ & ${\textstyle \mu_{ij} =  \alpha_i-\alpha_j}$\\
$\alt$ & ${\textstyle \mu_{ij} =  \gamma_{ij}}$ &
${\textstyle \gamma_{ij} = -\gamma_{ji}}$\\
{Differential Effects}\\
$DE_\theta$ & ${\textstyle \mu_{ij} =  \alpha_i \cos\theta
+\alpha_j \sin\theta}$ & ${\textstyle \theta \in [0,\pi)}$\\
$DE$ & ${\textstyle \mu_{ij} =  \alpha_i \cos\theta
+\alpha_j \sin\theta}$ & \\
\hline
\end{tabular}
}
\end{center}

Nicely placed tables with captions, numbers and labels like table \ref{tab:symgrporbits}, can be produced with \verb+\begin{table}+ and \verb+\end{table}+.


\begin{table}[tb]
\caption{Orbits of $S_n$ in $\Omega^3$ (with $n>3$)
and in $\Omega^4$ (with $n>4$)}
\label{tab:symgrporbits}
\begin{center}
\begin{tabular}{lr||lr}
\multicolumn{2}{c}{Orbits in $\Omega^3$}
&
\multicolumn{2}{c}{Orbits in $\Omega^4$} \\
Orbit type & multiplicity & Orbit type& multiplicity\\
\hline
$\{(i,i,i)\}$ & 1 & $\{(i,i,i,i)\}$ & 1 \\
$\{(i,i,j)\}$ & 3 & $\{(i,i,i,j)\}$ & 4 \\
$\{(i,j,k)\}$ & 1 & $\{(i,i,j,j)\}$ & 3 \\
&   & $\{(i,i,j,k)\}$& 6 \\
&   & $\{(i,j,k,l)\}$& 1 \\
\hline
Total        &  5 & Total        &  15
\end{tabular}
\end{center}
\end{table}


\section{Lists}

Devotees of the bullet point:
\begin{itemize}
\item Should use the \verb+\begin{itemize}+ command to start a bulleted list.
\item Should use the \verb+\item+ command to add items to the list.
\item Should use the \verb+\end{itemize}+ command to end a bulleted list.
\end{itemize}

\noindent Numbered lists can be useful
\begin{enumerate}
\item because I say so.
\item because they break op the visual flatness of a page.
\item because they can be nested which is useful for
\begin{enumerate}
\item examination questions.
\item algorithms.
\item er.
\end{enumerate}
\end{enumerate}

\section{Bibliography and references}

Quite nice bibliography creating facilities are available using the BibTex program. Basically you create a separate file containing all your references, cite them in the document using standard commands and then place a command at the end of you document to create the reference list --- only references that you actually cited will appear on this list.

Here are some example citations: \cite{Billon2004}.

\section{Table of contents}

\tableofcontents
\nocite{*}
\bibliographystyle{apalike}
\bibliography{A_Minor_Research_Project} 

\end{document}